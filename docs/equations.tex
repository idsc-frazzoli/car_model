%%%%%%%%%%%%%%%%%%%%%%%%%%%%%%%%%%%%%%%%%
% Short Sectioned Assignment
% LaTeX Template
% Version 1.0 (5/5/12)
%
% This template has been downloaded from:
% http://www.LaTeXTemplates.com
%
% Original author:
% Frits Wenneker (http://www.howtotex.com)
%
% License:
% CC BY-NC-SA 3.0 (http://creativecommons.org/licenses/by-nc-sa/3.0/)
%
%%%%%%%%%%%%%%%%%%%%%%%%%%%%%%%%%%%%%%%%%

%----------------------------------------------------------------------------------------
%	PACKAGES AND OTHER DOCUMENT CONFIGURATIONS
%----------------------------------------------------------------------------------------

\documentclass[paper=a4, fontsize=11pt]{scrartcl} % A4 paper and 11pt font size

\usepackage[T1]{fontenc} % Use 8-bit encoding that has 256 glyphs
\usepackage{fourier} % Use the Adobe Utopia font for the document - comment this line to return to the LaTeX default
\usepackage[english]{babel} % English language/hyphenation
\usepackage{amsmath,amsfonts,amsthm} % Math packages
\usepackage{graphicx}  

\usepackage{lipsum} % Used for inserting dummy 'Lorem ipsum' text into the template

\usepackage{sectsty} % Allows customizing section commands
\allsectionsfont{\centering \normalfont\scshape} % Make all sections centered, the default font and small caps

\usepackage{fancyhdr} % Custom headers and footers
\pagestyle{fancyplain} % Makes all pages in the document conform to the custom headers and footers
\fancyhead{} % No page header - if you want one, create it in the same way as the footers below
\fancyfoot[L]{} % Empty left footer
\fancyfoot[C]{} % Empty center footer
\fancyfoot[R]{\thepage} % Page numbering for right footer
\renewcommand{\headrulewidth}{0pt} % Remove header underlines
\renewcommand{\footrulewidth}{0pt} % Remove footer underlines
\setlength{\headheight}{13.6pt} % Customize the height of the header

\numberwithin{equation}{section} % Number equations within sections (i.e. 1.1, 1.2, 2.1, 2.2 instead of 1, 2, 3, 4)
\numberwithin{figure}{section} % Number figures within sections (i.e. 1.1, 1.2, 2.1, 2.2 instead of 1, 2, 3, 4)
\numberwithin{table}{section} % Number tables within sections (i.e. 1.1, 1.2, 2.1, 2.2 instead of 1, 2, 3, 4)

\setlength\parindent{0pt} % Removes all indentation from paragraphs - comment this line for an assignment with lots of text

%----------------------------------------------------------------------------------------
%	TITLE SECTION
%----------------------------------------------------------------------------------------

\newcommand{\horrule}[1]{\rule{\linewidth}{#1}} % Create horizontal rule command with 1 argument of height

\title{	
\normalfont \normalsize 
\textsc{ETH Zurich, IDSC} \\ [25pt] % Your university, school and/or department name(s)
\horrule{0.5pt} \\[0.4cm] % Thin top horizontal rule
\huge Car model Equations \\ % The assignment title
\horrule{2pt} \\[0.5cm] % Thick bottom horizontal rule
}

\author{Edo Jelavic} % Your name

\date{\normalsize\today} % Today's date or a custom date

\begin{document}

\maketitle % Print the title



\section{Four Wheeled Car Model} \label{FourWheelModel}

Four wheeled car model is built upon the schematic depicted in \ref{4wheelCar}.

\begin{figure}[h!]
	\centering
	\includegraphics[width=0.6\textwidth]{drawings/4wheelCar.png}
	\caption{4 wheeled car model}
	\label{4wheelCar}
\end{figure}

In Figure \ref{4wheelCar}, superscript $1L$ refers to the front left wheel, $1R$ refers to front right wheel and similarly $2L$ and $2R$ refer to rear left and right wheel, respectively. Subscripts $x$ and $y$ refer to longitudinal or lateral forces. Capital letter $F$ denotes a force that is tied to the car coordinate frame (red frames and forces painted red in Figure \ref{4wheelCar}) while the lower case letters $f$ denote forces in the tire coordinate frame. Four wheeled model is described by following equations:

\begin{align}
\dot{{U_x}} &  = \frac{1}{m} \left( F_x^{1L} + F_x^{1R} + F_x^{2L} +F_x^{2R} \right) + U_yr \label{4wheelCarEq1}      \\
\dot{U_y}   & = \frac{1}{m} \left(F_y^{1L} + F_y^{1R} + F_y^{2L} + F_y^{2R}  \right)  - U_xr \\
\dot{r}  & = \frac{1}{I_z}  \left( l_F \left(F_y^{1L} + F_y^{1R} \right) -l_R \left(F_y^{2L} + F_y^{2R} \right) + \frac{w}{2}\left(F_x^{1R} + F_x^{2R} - F_x^{1L} - F_x^{2L} \right)    \right) \\
\dot{\psi}  & = r \\
\dot{x}  & = U_x\cos\psi - U_y\sin\psi \\
\dot{y}  & = U_x\sin\psi + U_y\cos\psi \\
\dot{\omega}^{1L}  & = \frac{1}{I_{\omega}}\left( \hat{T}^{1L} - f_x^{1L}R\right)  \\
\dot{\omega}^{1R}  & = \frac{1}{I_{\omega}}\left( \hat{T}^{1R} - f_x^{1R}R\right) \\ 
\dot{\omega}^{2L}  & = \frac{1}{I_{\omega}}\left( \hat{T}^{2L} - f_x^{2L}R\right)  \\
\dot{\omega}^{2R}  & = \frac{1}{I_{\omega}}\left( \hat{T}^{2R} - f_x^{2R}R\right) \label{4wheelCarEq10} \\ 
\hat{T}^j& = \begin{cases}   T^j, \qquad \omega^j > 0 \\
f_x^jR, \qquad \omega^j = 0 
\end{cases}, \text{ where } \quad j \in \{1L, 1R, 2L, 2R \} \label{discontinuous}
\end{align}

In equations \ref{4wheelCarEq1} - \ref{4wheelCarEq10} $w$ denotes width of the car. Superscript $1$ denotes the front wheel and superscript $2$ denotes the rear wheel. Subscript $x$ denotes longitudinal direction and subscript $y$ denotes lateral direction. For example, $f^1_x$ is a force on the front tire in the longitudinal direction (in the tire frame). $U_x$ and $U_y$ denote longitudinal and lateral velocities of the car, respectively. Yaw rate is denoted with $r$ and $\psi$ is the yaw angle (or heading angle). 
$m$ is a mass of the car.
$I_z$ is a moment of inertia around the z-axis.
$I_\omega$ is the wheel moment of inertia around axis of rotation of wheel.
$l_F$ is a distance between Center of Gravity (COG) and the front axle, $l_R$ is the distance between COG and the rear axle. $w$ is the distance from left to the right wheel. $R$ is the radius of the wheels. The rotational velocity of the wheels is denoted with $\omega$ and $\mu$ is a friction coefficient between the tire and the surface. Steering angle is denoted with $\delta$ and $T$ denotes torque applied to the wheels (those are inputs into the system). Lateral and longitudinal forces in the tire frame are calculated as follows:

\begin{align}
\mu^j & =D\sin(C\arctan(Bs^j)) \label{4wheelPacejkaEq1}  \\
\mu_i^j & = -\frac{s_i^j}{s^j}\mu^j   \\
f_i^j & = \mu f_{z}^j \mu_i^j \label{lateralLongitudinal}  \\
s_x^j & = \frac{v_x^j - \omega^jR}{\omega^jR}   \\
s_y^j & = \left( 1 + s_x^j\right) \frac{v_y^j}{v_x^j}  \\ 
s^j & = \sqrt{ (s_x^j)^2 + (s_y^j)^2 }   \\
where \quad &  i \in\left\{ {x, y}\right\}, j \in \left\{ {1L,1R,2L, 2R}\right\} \label{4wheelPacejkaEq6}
\end{align}

Relationship between forces in the car frame (red frame in Figure \ref{4wheelCar}) and forces in the tire frame is given by:

\begin{align}
\begin{bmatrix}
F^j_x  \label{forcesConversion1}\\
F^j_y
\end{bmatrix} & = \begin{bmatrix}
\cos\delta & -\sin\delta \\
\sin\delta & \cos\delta
\end{bmatrix} \begin{bmatrix}
f^j_x \\
f^j_y
\end{bmatrix}
\text{where}  \quad j \in \left\{ {1L,1R}\right\} \\
\begin{bmatrix}
F^j_x \\
F^j_y
\end{bmatrix} & = \begin{bmatrix}
f^j_x \\
f^j_y
\end{bmatrix}
\text{where}  \quad j \in \left\{ {2L,2R}\right\} \label{forcesConversion3} \\
\end{align}

Similar relations hold for velocities as well:

\begin{align}
\begin{bmatrix}
v^{1L}_x \\
v^{1L}_y
\end{bmatrix} & = \begin{bmatrix}
\cos\delta & \sin\delta \\
-\sin\delta & \cos\delta
\end{bmatrix} \begin{bmatrix}
U_x - r\frac{w}{2} \\
U_y + r l_F
\end{bmatrix} \\
\begin{bmatrix}
v^{1R}_x \\
v^{1R}_y
\end{bmatrix} & = \begin{bmatrix}
\cos\delta & \sin\delta \\
-\sin\delta & \cos\delta
\end{bmatrix} \begin{bmatrix}
U_x + r\frac{w}{2} \\
U_y + r l_F
\end{bmatrix} \\
\begin{bmatrix}
v^{2L}_x \\
v^{2L}_y
\end{bmatrix} & = \begin{bmatrix}
U_x - r\frac{w}{2} \\
U_y - r l_R
\end{bmatrix}\\
\begin{bmatrix}
v^{2R}_x \\
v^{2R}_y
\end{bmatrix} & = \begin{bmatrix}
U_x + r\frac{w}{2} \\
U_y - r l_R
\end{bmatrix}
\end{align}

Still missing, are the expressions for computing vertical forces $F^j_z$ (note that $F^j_z = f^j_z$). To obtain expressions for these forces a weight transfer model has to be derived. It all starts by looking at the car confined to the $y-z$ plane. Rear view of the car in the $y-z$ plane is shown in Figure \ref{4wheelYZplane}.

\begin{figure}[h!]
	\centering
	\includegraphics[width=0.7\textwidth]{drawings/4wheelYZplane.png}
	\caption{4 wheeled car model, back view}
	\label{4wheelYZplane}
\end{figure}

Looking at the Figure \ref{4wheelYZplane} one obtains the first equation of four equations in total. The first equation is expressed in the equation \ref{YZtorqueBalance}; it is merely a balance of torques around the center of gravity. $h$ denotes the distance between COG and the ground.

\begin{equation}
\left(F^{1L}_z+F^{2L}_z \right)\frac{w}{2} + h\left(F^{1L}_y + F^{1R}_y + F^{2L}_y + F^{2R}_y \right) = \left(F^{1R}_z+F^{2R}_z \right)\frac{w}{2} \label{XZtorqueBalance}
\end{equation} 

Second equation is obtained by looking at the torque balance in $x-z$ plane. Side view of the car is shown in Figure \ref{4wheelXZplane}.

\begin{figure}[h!]
	\centering
	\includegraphics[width=0.7\textwidth]{drawings/4wheelXZplane.png}
	\caption{4 wheeled car model, side view}
	\label{4wheelXZplane}
\end{figure}

Equation of torque balance in $x-z$ plane is given by:

\begin{equation}
\left(F^{2R}_z+F^{2L}_z \right)l_R + h\left(F^{1L}_x + F^{1R}_x + F^{2L}_x + F^{2R}_x\right) = \left(F^{1R}_z+F^{1L}_z \right)l_F \label{YZtorqueBalance}
\end{equation}

The third equation is given by:

\begin{equation}
F^{1L}_z+F^{1R}_z + F^{2L}_z + F^{2R}_z = mg\label{gravityBalance}
\end{equation} 

Where $m$ is mass and $g$ is gravitational acceleration. Finally, the last equation arises from geometrical considerations. In Figure \ref{deflectedPlane}, a deflected $x-y$ plane is shown. It is assumed that all the rotations happen around the COG.

\begin{figure}[h!]
	\centering
	\includegraphics[width=0.7\textwidth]{drawings/deflectedPlane.png}
	\caption{4 wheeled car model, deflected $x-y$ plane amid weight transfer}
	\label{deflectedPlane}
\end{figure}

From Figure \ref{deflectedPlane} it can be seen that deflection on the wheel $2R$ has to be equal to the deflection at the diagonally opposite wheel $1L$. Hence, for the deflections $z$, it has to hold:

\begin{align}
z^{1L} + z^{2R} &= 0 \label{deflections1}\\
z^{1R} + z^{2L} &= 0 \label{deflections2}
\end{align}

If one sums up the equations \ref{deflections1} and \ref{deflections2}, one obtains:

\begin{equation}
z^{1L} + z^{2R} = z^{1R} + z^{2L} \label{deflections} 
\end{equation}

It is assumed that there is a spring on each wheel. Therefore, the following equation holds:

\begin{equation}
F^j_z =c^jz^j, where \quad j \in \{1L,1R,2L,2R \} \label{spring} 
\end{equation}

In equation \ref{spring}, $z$ are deflections and $c$ are spring constants. Combining equations \ref{spring} and \ref{deflections} and using a simplification $c^i = c^j, \forall i,j \in \{1L,1R,2L,2R \}$, one gets the fourth equation:

\begin{equation}
F_z^{1L} + F_z^{2R} = F_z^{1R} + F_z^{2L} \label{forcesBalance} 
\end{equation}

By plugging in the expressions for lateral and longitudinal forces ($F_x$, $F_y$) from equation \ref{lateralLongitudinal} and equations \ref{forcesConversion1} - \ref{forcesConversion3}  into equations \ref{YZtorqueBalance}, \ref{XZtorqueBalance}, \ref{gravityBalance} and \ref{forcesBalance}, one obtains expressions for vertical forces $F_z$. These are given by the following set of equations:

\begin{align}
F^{1L}_z &= mg\frac{BG - CF + BH - DF - CH + DG}{den} \\
F^{1R}_z &= mg\frac{AG - EC + AH - ED + CH - DG}{den} \\
F^{2L}_z &= mg\frac{AF - EB + AH - ED + BH - DF}{den} \\
F^{1R}_z &= mg\frac{AF - EB - AG + EC - BG + CF}{den} \\
\end{align}

where:

\begin{align}
den &= 2(AF - EB - AG + EC + BH - DF - CH + DG) \\
A &= -\frac{w}{2} - C_1 - C_2 \\
B &= \frac{w}{2} - C_3 - C_4 \\
C &= -\frac{w}{2} - C_5 \\
D &= \frac{w}{2} - C_6 \\
E &= K_1 - K_2 - l_F \\
F &= K_3 - K_4 - l_F \\
G &= K_5 + l_R \\
H &= K_6 + l_R \\
C_1 &= -\mu \mu^{1L}_x h \sin(\delta) \\
C_2 &= -\mu \mu^{1L}_y h \cos(\delta) \\
C_3 &= -\mu \mu^{1R}_x h \sin(\delta) \\
C_4 &= -\mu \mu^{1R}_y h \cos(\delta) \\
C_5 &= -\mu \mu^{2L}_y h  \\
C_6 &= -\mu \mu^{2R}_y h  \\
K_1 &= \mu \mu^{1L}_x h \cos(\delta) \\
K_2 &= \mu \mu^{1L}_y h \sin(\delta) \\
K_3 &= \mu \mu^{1R}_x h \cos(\delta) \\
K_4 &= \mu \mu^{1R}_y h \sin(\delta) \\
K_5 &= \mu \mu^{2L}_x h \\
K_6 &= \mu \mu^{2R}_x h \\
\end{align}

This concludes the derivation of the four-wheeled car model.


Should any errors or inconsistencies be found, feel free to correct them ot make the author aware of them.



%----------------------------------------------------------------------------------------

\end{document}
